\documentclass[]{article}
\usepackage{lmodern}
\usepackage{amssymb,amsmath}
\usepackage{ifxetex,ifluatex}
\usepackage{fixltx2e} % provides \textsubscript
\ifnum 0\ifxetex 1\fi\ifluatex 1\fi=0 % if pdftex
  \usepackage[T1]{fontenc}
  \usepackage[utf8]{inputenc}
\else % if luatex or xelatex
  \ifxetex
    \usepackage{mathspec}
  \else
    \usepackage{fontspec}
  \fi
  \defaultfontfeatures{Ligatures=TeX,Scale=MatchLowercase}
\fi
% use upquote if available, for straight quotes in verbatim environments
\IfFileExists{upquote.sty}{\usepackage{upquote}}{}
% use microtype if available
\IfFileExists{microtype.sty}{%
\usepackage{microtype}
\UseMicrotypeSet[protrusion]{basicmath} % disable protrusion for tt fonts
}{}
\usepackage[margin=1in]{geometry}
\usepackage{hyperref}
\hypersetup{unicode=true,
            pdftitle={Creating paper frontpage with author affiliations in R},
            pdfauthor={Vincent Pellissier},
            pdfborder={0 0 0},
            breaklinks=true}
\urlstyle{same}  % don't use monospace font for urls
\usepackage{color}
\usepackage{fancyvrb}
\newcommand{\VerbBar}{|}
\newcommand{\VERB}{\Verb[commandchars=\\\{\}]}
\DefineVerbatimEnvironment{Highlighting}{Verbatim}{commandchars=\\\{\}}
% Add ',fontsize=\small' for more characters per line
\usepackage{framed}
\definecolor{shadecolor}{RGB}{248,248,248}
\newenvironment{Shaded}{\begin{snugshade}}{\end{snugshade}}
\newcommand{\KeywordTok}[1]{\textcolor[rgb]{0.13,0.29,0.53}{\textbf{#1}}}
\newcommand{\DataTypeTok}[1]{\textcolor[rgb]{0.13,0.29,0.53}{#1}}
\newcommand{\DecValTok}[1]{\textcolor[rgb]{0.00,0.00,0.81}{#1}}
\newcommand{\BaseNTok}[1]{\textcolor[rgb]{0.00,0.00,0.81}{#1}}
\newcommand{\FloatTok}[1]{\textcolor[rgb]{0.00,0.00,0.81}{#1}}
\newcommand{\ConstantTok}[1]{\textcolor[rgb]{0.00,0.00,0.00}{#1}}
\newcommand{\CharTok}[1]{\textcolor[rgb]{0.31,0.60,0.02}{#1}}
\newcommand{\SpecialCharTok}[1]{\textcolor[rgb]{0.00,0.00,0.00}{#1}}
\newcommand{\StringTok}[1]{\textcolor[rgb]{0.31,0.60,0.02}{#1}}
\newcommand{\VerbatimStringTok}[1]{\textcolor[rgb]{0.31,0.60,0.02}{#1}}
\newcommand{\SpecialStringTok}[1]{\textcolor[rgb]{0.31,0.60,0.02}{#1}}
\newcommand{\ImportTok}[1]{#1}
\newcommand{\CommentTok}[1]{\textcolor[rgb]{0.56,0.35,0.01}{\textit{#1}}}
\newcommand{\DocumentationTok}[1]{\textcolor[rgb]{0.56,0.35,0.01}{\textbf{\textit{#1}}}}
\newcommand{\AnnotationTok}[1]{\textcolor[rgb]{0.56,0.35,0.01}{\textbf{\textit{#1}}}}
\newcommand{\CommentVarTok}[1]{\textcolor[rgb]{0.56,0.35,0.01}{\textbf{\textit{#1}}}}
\newcommand{\OtherTok}[1]{\textcolor[rgb]{0.56,0.35,0.01}{#1}}
\newcommand{\FunctionTok}[1]{\textcolor[rgb]{0.00,0.00,0.00}{#1}}
\newcommand{\VariableTok}[1]{\textcolor[rgb]{0.00,0.00,0.00}{#1}}
\newcommand{\ControlFlowTok}[1]{\textcolor[rgb]{0.13,0.29,0.53}{\textbf{#1}}}
\newcommand{\OperatorTok}[1]{\textcolor[rgb]{0.81,0.36,0.00}{\textbf{#1}}}
\newcommand{\BuiltInTok}[1]{#1}
\newcommand{\ExtensionTok}[1]{#1}
\newcommand{\PreprocessorTok}[1]{\textcolor[rgb]{0.56,0.35,0.01}{\textit{#1}}}
\newcommand{\AttributeTok}[1]{\textcolor[rgb]{0.77,0.63,0.00}{#1}}
\newcommand{\RegionMarkerTok}[1]{#1}
\newcommand{\InformationTok}[1]{\textcolor[rgb]{0.56,0.35,0.01}{\textbf{\textit{#1}}}}
\newcommand{\WarningTok}[1]{\textcolor[rgb]{0.56,0.35,0.01}{\textbf{\textit{#1}}}}
\newcommand{\AlertTok}[1]{\textcolor[rgb]{0.94,0.16,0.16}{#1}}
\newcommand{\ErrorTok}[1]{\textcolor[rgb]{0.64,0.00,0.00}{\textbf{#1}}}
\newcommand{\NormalTok}[1]{#1}
\usepackage{graphicx,grffile}
\makeatletter
\def\maxwidth{\ifdim\Gin@nat@width>\linewidth\linewidth\else\Gin@nat@width\fi}
\def\maxheight{\ifdim\Gin@nat@height>\textheight\textheight\else\Gin@nat@height\fi}
\makeatother
% Scale images if necessary, so that they will not overflow the page
% margins by default, and it is still possible to overwrite the defaults
% using explicit options in \includegraphics[width, height, ...]{}
\setkeys{Gin}{width=\maxwidth,height=\maxheight,keepaspectratio}
\IfFileExists{parskip.sty}{%
\usepackage{parskip}
}{% else
\setlength{\parindent}{0pt}
\setlength{\parskip}{6pt plus 2pt minus 1pt}
}
\setlength{\emergencystretch}{3em}  % prevent overfull lines
\providecommand{\tightlist}{%
  \setlength{\itemsep}{0pt}\setlength{\parskip}{0pt}}
\setcounter{secnumdepth}{0}
% Redefines (sub)paragraphs to behave more like sections
\ifx\paragraph\undefined\else
\let\oldparagraph\paragraph
\renewcommand{\paragraph}[1]{\oldparagraph{#1}\mbox{}}
\fi
\ifx\subparagraph\undefined\else
\let\oldsubparagraph\subparagraph
\renewcommand{\subparagraph}[1]{\oldsubparagraph{#1}\mbox{}}
\fi

%%% Use protect on footnotes to avoid problems with footnotes in titles
\let\rmarkdownfootnote\footnote%
\def\footnote{\protect\rmarkdownfootnote}

%%% Change title format to be more compact
\usepackage{titling}

% Create subtitle command for use in maketitle
\newcommand{\subtitle}[1]{
  \posttitle{
    \begin{center}\large#1\end{center}
    }
}

\setlength{\droptitle}{-2em}
  \title{Creating paper frontpage with author affiliations in R}
  \pretitle{\vspace{\droptitle}\centering\huge}
  \posttitle{\par}
  \author{Vincent Pellissier}
  \preauthor{\centering\large\emph}
  \postauthor{\par}
  \date{}
  \predate{}\postdate{}


\begin{document}
\maketitle

Creating the first page of a paper is seemingly easy, you just collect
the name and affiliations of everyone, list the authors in whatever
order you or your supevisor decide, you list and number the affiliations
in the order they appear and you report these numbers (often as
upperscript) after each author names. However, if your paper have a long
list of authors it might become tedious to do it manually. Even worse,
there is always this one author in the middle of the list that suddenly
add an affilition (granted, it might just be because you spent a lot of
time finishing the paper, and then everyone moved). Anyway, it will then
requires you to manually shift every single number after this authors.

While I was facing this exact problem a few weeks ago, I decided to be
lazy and write an R package to do it, instead of renumbering every
single one of my 40 co-authors.

\subsection{Installing the package
frontpage}\label{installing-the-package-frontpage}

As the package is not distributed by CRAN right now, you need to install
it from GitHub

\begin{Shaded}
\begin{Highlighting}[]
\KeywordTok{install.packages}\NormalTok{(}\StringTok{'devtools'}\NormalTok{)}
\NormalTok{devtools}\OperatorTok{::}\KeywordTok{install_github}\NormalTok{(}\StringTok{'vpellissier/frontpage'}\NormalTok{)}
\end{Highlighting}
\end{Shaded}

\subsection{Creating the author list}\label{creating-the-author-list}

The author list with affiliations is stored as a CSV file, because it is
conveniently modified using any spreadsheet tool. The CSV file should
contain only one author per line, with the full name of the author in
the first column, and every affiliations of that author in the following
column (one affiliation per column only). The CSV can have header or not
(i.e.~the fist row contain column names or is simply the author). If you
are using Excel to create the CSV, you should save using the type 'CSV
(Comma delimited) (*.csv)`and not any of the other two options ('CSV
(Macintosh)' or `CSV (MS-DOS)').

I included a small example file in the package:

\begin{Shaded}
\begin{Highlighting}[]
\KeywordTok{shell.exec}\NormalTok{(}\DataTypeTok{file =} \KeywordTok{file.path}\NormalTok{(}\KeywordTok{path.package}\NormalTok{(}\StringTok{'frontpage'}\NormalTok{), }\StringTok{"extdata/example.csv"}\NormalTok{))}
\end{Highlighting}
\end{Shaded}

\subsection{Creating the paper first
page}\label{creating-the-paper-first-page}

Creating the paper first page is really straigtfoward when you have your
CSV file. First, you create your author list in R from your CSV file:

\begin{Shaded}
\begin{Highlighting}[]
\NormalTok{authors <-}\StringTok{ }\KeywordTok{read.csv2}\NormalTok{(}\KeywordTok{file.path}\NormalTok{(}\KeywordTok{path.package}\NormalTok{(}\StringTok{'frontpage'}\NormalTok{), }\StringTok{"extdata/example.csv"}\NormalTok{), }\DataTypeTok{header=}\NormalTok{ T)}
\end{Highlighting}
\end{Shaded}

Here, the first row of example.csv contains column name, so the
\texttt{header} option is set to \texttt{TRUE}

Then, you can create your frontpage and save it wherever you want (here,
in a temporary folder):

\begin{Shaded}
\begin{Highlighting}[]
\NormalTok{temp <-}\StringTok{ }\KeywordTok{dir.create}\NormalTok{(}\KeywordTok{tempdir}\NormalTok{())}
\KeywordTok{frontpage}\NormalTok{(author, }\KeywordTok{file.path}\NormalTok{(temp, }\StringTok{'frontpage1.docx'}\NormalTok{))}
\end{Highlighting}
\end{Shaded}

And all that is left to do is write the rest of your paper in that doc
file!

\begin{Shaded}
\begin{Highlighting}[]
\KeywordTok{shell.exec}\NormalTok{(}\KeywordTok{file.path}\NormalTok{(temp, }\StringTok{'frontpage1.docx'}\NormalTok{))}
\end{Highlighting}
\end{Shaded}


\end{document}
